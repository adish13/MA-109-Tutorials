\documentclass[handout]{beamer}
\usepackage[utf8]{inputenc}
\usepackage{ stmaryrd }
\usepackage{tikz}

\usepackage{physics}
\usepackage{hyperref}
\usepackage{amsmath}
\usepackage{amsthm}
\usepackage{amssymb}
\usetheme{Madrid}
% \mode<presentation>{}
\usecolortheme{default}
\usepackage{mathtools}
\DeclarePairedDelimiter\ceil{\lceil}{\rceil}
\DeclarePairedDelimiter\floor{\lfloor}{\rfloor}
\newcommand{\cosec}{\operatorname{cosec}}

%------------------------------------------------------------
%This block of code defines the information to appear in the
%Title page
\title[MA109 Calculus-I] %optional
{MA109 Calculus-I}

\subtitle{D4-T6 Tutorial 4}

\author[Adish Shah] % (optional)
{Adish Shah}



\date[29th December 2021] % (optional)
{29th December 2021}



%End of title page configuration block
%------------------------------------------------------------



%------------------------------------------------------------
%The next block of commands puts the table of contents at the 
%beginning of each section and highlights the current section:

\AtBeginSection[]
{
  \begin{frame}
    \frametitle{Table of Contents}
    \tableofcontents[currentsection]
  \end{frame}
}
%------------------------------------------------------------


\begin{document}

%The next statement creates the title page.
\frame{\titlepage}

\begin{frame}
	\frametitle{{Fundamental Theorem of Calculus : Part 1}}
	\begin{theorem}[Fundamental Theorem of Calculus : Part 1]
		Let $f$ be integrable on $[a, b]$. For $x \in [a,b]$,define\\
		\[F(x) := \int_{a}^{b}f(t)dt\] 
	Then $F$ is continuous on $[a,b]$. Morover, if $f$ is continuous at $c \in [a,b]$,
	then $F$ is differentiable at $c$, and $F'(c) = f(c)$
	\end{theorem}

\end{frame}
\begin{frame}
	\frametitle{{Fundamental Theorem of Calculus : Part 2}}
	\begin{theorem}[Fundamental Theorem of Calculus : Part 2]
		Let $f:[a, b] \to \mathbb{R}$ be a differentiable function such that\\
		$f'$ is integrable on $[a, b],$. Then\\
		\[\int_{a}^{b}f'(x)dx = f(b)-f(a)\]
	\end{theorem}
\end{frame}

%---------------------------------------------------------
%Changing visivility of the text
\begin{frame}
	\frametitle{5)}
	The given function is reimann integrable as it is a monotonically increasing function. \\
	Let $P_{n}$ be the partition of [0,2] into $2 \times 2^{n}$ parts.

	Then $U(P_{n},f) = 3$ and

	\[L(P_{n}) = \sum_{i=1}^{n} m_i(x_i-x_{i-1}) =  1 + 1 \times \frac{1}{2^{n}} + 2 \times \frac{2^{n}-1}{2^{n}} \rightarrow 3 \]
	as $n \rightarrow \infty$.

	Thus, $\int_{0}^{2}f(x)dx = 3$
\end{frame}

\begin{frame}
	\frametitle{6)}
	(a)We know that $L(P)\leq \int_{a}^{b}f(x)dx \leq U(P)$,
	\begin{align*}
		L(P)                           & =\sum_{i=1}^{n} m_i(x_i-x_{i-1}) \\
		\Rightarrow L(P)               & \geq 0                           \\
		\Rightarrow \int_{a}^{b}f(x)dx & \geq 0
	\end{align*}
	Since $m_i\geq 0$  $\forall i$. Further, if $f$ is continuous let $F(x)$ be defined by $F(x)=\int_{a}^{x}f(t)dt$, then from FTC
	$$F'(x)=f(x)\geq 0\forall x\in [a,b]$$
	Now we know that $F'(x)\geq 0$, $F(b)=F(a)=0\Rightarrow$ $F(x)=0\forall x\in[a,b]\Rightarrow f(x)=0\forall x\in[a,b]$
\end{frame}

\begin{frame}
	\frametitle{6) continued}
	(b) Take $f(x)=0$ if $x\neq \frac{a+b}{2}$, $f(\frac{a+b}{2})=1$. Then this function is Riemann integrable and
	$$\int_{a}^{b}f(x)dx =0 $$
\end{frame}

\begin{frame}
	\frametitle{7)}
	i) Note that \\
	\[S_n = \dfrac{1}{n^{5/2}}\displaystyle\sum_{i=1}^{n}i^{3/2} = \sum_{i=1}^{n}\left(\dfrac{i}{n}\right)^{3/2}\left(\dfrac{i}{n} - \dfrac{i-1}{n}\right).\]
	Define $f:[0, 1] \to \mathbb{R}$ by $f(x) := \frac{2}{5}x^{5/2}.$ Then, we have that $f'(x) = x^{3/2}.$\\
	As $f'$ is continuous and bounded, it is (Riemann) integrable. \\
	For $n \in \mathbb{N},$ let $P_n := \{0, 1/n, \ldots, n/n\}$ and $t_i := i/n$ for $i = 1, 2, \ldots, n.$\\
	Then, $S_n = R(f',P_n,t).$ Since $||(P_n)|| = \frac{1}{n} \to 0,$ it follows that
	\[R(f',P_n,t) \to \int_{0}^{1} x^{3/2} dx = \int_{0}^{1} f'(x) dx. \]
	By the Fundamental Theorem of Calculus (Part 2), we have it that
	\[\lim_{n\to \infty}S_n = \int_{0}^{1} f'(x) dx = f(1) - f(0) = \dfrac{2}{5}.\]

\end{frame}


\begin{frame}
	\frametitle{7)}
	(iii) Note that \\
	\[S_n = \sum_{i=1}^{n}\dfrac{1}{\sqrt{in + n^2}} = \sum_{i=1}^{n}\dfrac{1}{\sqrt{\left(\frac{i}{n}\right) + 1}}\left(\frac{i}{n} - \frac{i-1}{n}\right) .\]
	Define $f:[0, 1] \to \mathbb{R}$ by $f(x) := 2\sqrt{x + 1}.$ Then, we have that $f'(x) = \frac{1}{\sqrt{x+ 1}}.$\\
	As $f'$ is continuous and bounded, it is (Riemann) integrable. \\
	For $n \in \mathbb{N},$ let $P_n := \{0, 1/n, \ldots, n/n\}$ and $t_i := i/n$ for $i = 1, 2, \ldots, n.$\\
	Then, $S_n = R(f',P_n,t).$ Since $||(P_n)|| = \frac{1}{n} \to 0,$ it follows that
	\[R(f',P_n,t) \to \int_{0}^{1} \frac{1}{\sqrt{x+ 1}} dx = \int_{0}^{1} f'(x) dx. \]
	By the Fundamental Theorem of Calculus (Part 2), we have it that
	\[\lim_{n\to \infty}S_n = \int_{0}^{1} f'(x) dx = f(1) - f(0) = 2\sqrt{2} - 2.\]

\end{frame}

\begin{frame}
	\frametitle{7)}
	(iv) Note that \\
	\[S_n = \dfrac{1}{n}\sum_{i=1}^{n}\cos\left(\dfrac{i\pi}{n}\right) = \sum_{i=1}^{n}\cos\left(\dfrac{i\pi}{n}\right)\left(\frac{i}{n} - \frac{i-1}{n}\right) .\]
	Define $f:[0, 1] \to \mathbb{R}$ by $f(x) := \frac{1}{\pi}\sin(\pi x).$ Then, we have that $f'(x) = \cos(\pi x).$\\
	As $f'$ is continuous and bounded, it is (Riemann) integrable. \\
	For $n \in \mathbb{N},$ let $P_n := \{0, 1/n, \ldots, n/n\}$ and $t_i := i/n$ for $i = 1, 2, \ldots, n.$\\
	Then, $S_n = R(f',P_n,t).$ Since $||(P_n)|| = \frac{1}{n} \to 0,$ it follows that
	\[R(f',P_n,t) \to \int_{0}^{1} \cos(\pi x) dx = \int_{0}^{1} f'(x) dx. \]
	By the Fundamental Theorem of Calculus (Part 2), we have it that
	\[\lim_{n\to \infty}S_n = \int_{0}^{1} f'(x) dx = f(1) - f(0) = 0.\]

\end{frame}

\begin{frame}
	\frametitle{7)}
	(v) Note that \\
	\[S_n = \dfrac{1}{n}\left\{\sum_{i=1}^{n}\left(\frac{i}{n}\right) + \sum_{i=n+1}^{2n}\left(\frac{i}{n}\right)^{3/2} + \sum_{i=2n+1}^{3n}\left(\frac{i}{n}\right)^2\right\} .\]
	We shall find $\displaystyle\lim_{n\to \infty}S_n$ by finding the limits of the individual sums and showing that they all exist.\\
	\[S_{n} \rightarrow \int_{0}^{1}xdx + \int_{1}^{2} x^{3/2} dx + \int_{2}^{3} x^{2}dx = \frac{1}{2} + \frac{2}{5}(4\sqrt{2}-1) + \frac{19}{3}  \]
\end{frame}
\begin{frame}
	\frametitle{8 b) }
	Let $u$ and $v$ be differentiable functions defined on appropriate domains.\\
	Let $g$ be a continuous function. Define $G(x) := \displaystyle\int_{a}^{x} g(t) dt.$ Then $G'(x) = g(x),$ by Fundamental Theorem of Calculus (Part 1). Note that
	\[\int_{u(x)}^{v(x)} g(t) dt = \int_{a}^{v(x)} g(t) dt - \int_{a}^{u(x)} g(t) dt = G(v(x)) - G(u(x)).\]
	Thus, by the Chain Rule, one has
	\[\dfrac{d}{dx}\int_{u(x)}^{v(x)} g(t) dt = G'(v(x))v'(x) - G'(u(x))u'(x) = g(v(x))v'(x) - g(u(x))u'(x).\]
	We can now easily solve the question.
\end{frame}


\begin{frame}
	\frametitle{8b) continued}
	(i)\\
	Given, $F(x) = \displaystyle\int_{1}^{2x} \cos(t^2) dt $
	\begin{align*}
		\therefore \frac{dF}{dx} & = \cos\left((2x)^2\right)(2x)' - \cos(1)(1)' \\
		                         & = 2\cos(4x^2).
	\end{align*}
\end{frame}

\begin{frame}
	\frametitle{8b) continued}
	(ii)\\
	Given, $F(x) = \displaystyle\int_{0}^{x^2} \cos(t) dt $
	\begin{align*}
		\therefore \frac{dF}{dx} & = \cos\left(x^2\right)(x^2)' - \cos(0)(0)' \\
		                         & = 2x\cos(x^2).
	\end{align*}
\end{frame}
%---------------------------------------------------------
%Changing visivility of the text
\begin{frame}
	\frametitle{9)}
	Define $F:\mathbb{R} \to \mathbb{R}$ as
	\[F(a) := \int_{a}^{a+p} f(t) dt.\]
	If we show that $F$ is constant, then we are done.\\
	%Using domain additivity, we can write $F(a) = \displaystyle\int_{0}^{a+p} f(t)dt - \int_{0}^{a} f(t) dt.$\\
	As $f$ is a continuous, Fundamental Theorem of Calculus (Part 1) tells us that $F$ is differentiable everywhere. Using the result we had shown earlier, we have it that $F'(a) = f(a+p)\cdot 1 - f(a)\cdot 1 = 0.$\\
	As $F$ is defined on an interval ($\mathbb{R}$), we have it that $F$ is constant. \hfill $\blacksquare$
\end{frame}

\begin{frame}
	\frametitle{10)}
	\begin{align*}
		g(x) & = \dfrac{1}{\lambda}\int_{0}^{x} f(t)\sin \lambda(x - t) dt                                                                             \\
		     & = \dfrac{1}{\lambda}\int_{0}^{x} f(t) \left(\sin \lambda x\cos \lambda t - \cos \lambda x \sin \lambda t\right) dt                      \\
		     & = \frac{1}{\lambda}\sin\lambda x\int_{0}^{x} f(t)\cos \lambda t dt - \frac{1}{\lambda}\cos \lambda x \int_{0}^{x} f(t)\sin \lambda t dt \\
	\end{align*}
	Now, we can differentiate $g$ using product rule and Fundamental Theorem of Calculus (Part 1).
	\begin{align*}
		\therefore g'(x) %&= \cos\lambda x\int_{0}^{x} f(t)\cos \lambda t dt + \frac{1}{\lambda}f(x)\sin\lambda x \cos \lambda x dt + \sin \lambda x \int_{0}^{x} f(t)\sin \lambda t dt - \frac{1}{\lambda}f(x)\cos \lambda x \sin \lambda \\
		 & = \cos\lambda x\int_{0}^{x} f(t)\cos \lambda t dt + \sin \lambda x \int_{0}^{x} f(t)\sin \lambda t dt \\
	\end{align*}
\end{frame}


\begin{frame}
	\frametitle{10) continued}
	It is easy to verify that both $g(0)$ and $g'(0)$ are 0.\\
	We can differentiate $g'$ in a similar way and get,
	\begin{align*}
		g''(x)   & = -\lambda\sin\lambda x\int_{0}^{x} f(t)\cos \lambda t dt + f(x)\cos^2\lambda x + \lambda \cos \lambda x \int_{0}^{x} f(t)\sin \lambda t dt                    \\
		         & + f(x)\sin^2 \lambda x                                                                                                                                         \\
		         & = f(x) - \lambda^2\left(\dfrac{1}{\lambda}\int_{0}^{x} f(t) \left(\sin \lambda x\cos \lambda t - \cos \lambda x \sin \lambda t\right) dt\right)                \\
		         & = f(x) - \lambda^2g(x)                                                                                                                                         \\
		\implies & g''(x) + \lambda^2g(x) = f(x)                                                                                                                   & \blacksquare
	\end{align*}
\end{frame}

%---------------------------------------------------------

%Example of the \pause command
% \begin{frame}
%  in this slide \pause

% the text will be sounds partially visible \pause

% And finally everything will be there
% \end{frame}
%---------------------------------------------------------


% \section{Background of Shor's Algorithm}
% %---------------------------------------------------------
% \begin{frame}
% \frametitle{Background}
%     Let us look at the background of Shor's algorithm
%     \begin{itemize}
%       \item<1-> Quantum Algorithm, to find the prime factors of any given integer N
%       \item<2-> Named after a mathematician, Peter Shor who formulated the algorithm in 1994.
%       \item<3-> This algorithm cam factor a number N in $O((logN)^3)$ time and $O(logN)$ space.
%       \item<4-> This demonstrates that an integer factorization can be done on a quantum computer in polynomial time. 
%     \end{itemize}
% \end{frame}
% \begin{frame}
% \frametitle{RSA}
%     RSA is a popular encyrption technique that uses a public key N which is the product of two large prime numbers.
%     \begin{itemize}
%       \item<1-> One way to crack RSA encyrption is by factoring N, but, with classical algorithms, factoring becomes increasingly time-consuming as N grows larger.
%       \item<2-> More specifically, there is \textbf{no classical algorithm} known that can factor a number N in polynomial time.
%     \end{itemize}
%   \end{frame}
% \section{Grover's algorithm}

% \begin{frame}
%   Like many other quantum algorithms, Shor's algorithm is probabilistic. \pause
%   This means that it will give the correct answer with high probability, and the probability of success can be increased by performing more iterations.
% \end{frame}

% \begin{frame}
% Discrete Fourier Transform acts on a vector of complex numbers, $x_{0},x_{1},\dots , x_{N-1}$. of fixed length N, and outputs another vector of complex numbers,$y_{0},y_{1},\dots , y_{N-1}$ given by :
% $$y_{k} = \frac{1}{\sqrt{N}} \sum_{j=0}^{N-1}x_{j}e^{2 \pi ijk/N}$$ \pause
% The quantum Fourier Transform is defined in a similar way. The QFT on an orthonormal basis of vectors $\ket{0},\ket{1},\dots,\ket{N-1}$ is defined to be a linear operator with the following action on the basis state:
% $$\ket{j} \shortrightarrow  \frac{1}{\sqrt{N}} \sum_{j=0}^{N-1} \ket{k}e ^{2 \pi ijk/N}$$ \pause
% This can also be seen as :
% $$ \sum_{j=0}^{N-1} x_{j} \ket{j} = \sum_{k=0}^{N-1}y_{k} \ket{k}$$

% \end{frame}
%---------------------------------------------------------
% %Highlighting text
% \begin{frame}
% \frametitle{Introduction to Grover's algorithm}

% % In this slide, some important text will be
% % \alert{highlighted} because it's important.

% % Please, don't abuse it.

% % \begin{block}{Remark}
% % Sample text
% % \end{block}

% % \begin{alertblock}{Important theorem}
% % Sample text in red box
% % \end{alertblock}

% % \begin{examples}
% % Sample text in green box. The title of the block is ``Examples".
% % \end{examples}
% % \end{frame}

% Let us now look at another classical computing task that can be sped up using the superposition principle. \pause

% Our aim is to solve the needle in a haystack problem where we have to search for an element which satisfies a particular property, and it lies in a haystack of $N$ elements. \pause

% Classically, this would take $O(N)$ steps to do this task, but Grover's algorithm allows us to do it in $O(\sqrt{N})$ steps.
% \end{frame}

% \begin{frame}
%   \frametitle{Problem statement}
%   \begin{block}{Problem}
%     Suppose f(x) is a function from $\{0,1,\dots\}$ to $\{0,1\}$. f(a) = 1, only for one value of a. Find a. Assume that $N = 2^{n}$. This way you can work with n qubits.
%   \end{block}  
% \end{frame}

% \begin{frame}
%   \frametitle{Solution - Grover's Method}
%   \begin{center}
%     Let us define two new state vectors : \pause 
%   \end{center}
%   $$ \ket{\Psi_{0}} = \sum_{i=0}^{N-1} \frac{\ket{i}}{\sqrt{N}} $$ \pause
%   $$ \ket{e} = \sum_{i=0, i \neq a}^{N-1} \frac{\ket{i}}{\sqrt{N-1}} $$ \pause 

%   \begin{center}
%     Our goal is to ensure that we obtain $\ket{a}$ in the end. Notice that \pause
%     $$     \ket{\Psi_{0}} = \frac{\sqrt{N-1}\ket{e}+ \ket{a}}{\sqrt{N}}    $$ 
%     Thus $\ket{\Psi_{0}}$ lies in the subspace spanned by $\ket{e}$ and $\ket{a}$
%   \end{center}

% \end{frame}
% \begin{frame}
%   Note that, $\ket{\Psi_{0}}$ will be closer to $\ket{e}$ than $\ket{a}$. \pause
% %   \begin{figure}[htp]
% %     \centering
% %     \includegraphics[scale = 0.5]{grover.png}
% %   \end{figure}
% \end{frame}
% %---------------------------------------------------------
% \begin{frame}
%   We aim to take a vector $\ket{\Psi}$ which will initially be equal to $\ket{\Psi_{0}}$. \pause


%   Rotate it so that it ends up getting close to $\ket{a}$ \pause


%   First reflect $\Psi$ about $\ket{e}$ and then about $\ket{\Psi_{0}}$.
%   Reflecting about $\ket{e}$ can be considered as an oracle that performs the operation O
%   $$ \ket{x} \shortrightarrow (-1)^{f(x)}\ket{x}$$ \pause
%   The way to achieve this is to take an oracle that performs the operation 
%   $$ \ket{x}\ket{y} \shortrightarrow \ket{x} \ket{y \oplus f(x)}$$
%   Let $\ket{y}$ = $\frac{\ket{0}-\ket{1}}{\sqrt{2}}$ \pause
%   If x = a, $$\ket{x} \ket{y} \shortrightarrow - \ket{x} \ket{y} $$
%   Else, state will be preserved.

% \end{frame}

% \begin{frame}
%   \begin{block}{Remark}
%     It turns out mathematically, that Grover's operator G is 
%   $$G = (2\ket{\Psi_{0}}\bra{\Psi_{0}}- I)O$$ 
%   \end{block}
%   \pause 
% %   \begin{figure}[htp]
% %     \centering 
% %     \includegraphics[scale = 0.3]{Screenshot from 2021-07-19 22-30-03.png}
% %   \end{figure} \pause
%   Let $\ket{\Psi_{0}}$ make an angle $\theta$ with $\ket{e}$ in the two dimension subspace. Then,
%   $$ sin(\theta) = \frac{1}{\sqrt{N}} $$
% \end{frame}  
% %---------------------------------------------------------
% %Two columns
% \begin{frame}

% \begin{block}{Remark}
%   By induction, 
%   $$G^{k} \ket{\Psi_{0}} = cos((2k+1)\theta)\ket{e} + sin((2k+1)\theta)\ket{a}$$
% \end{block}
%  \pause 
% Applying G successively on $\ket{\Psi_{0}}$ rotates the vector by an angle $2\theta$ \pause 


% Goal is to make the angle between $\ket{\Psi}$ and $\ket{a} \le \theta$ \pause 


% Probability of collapsing will be $\ge cos(\theta) = \frac{\sqrt{N-1}}{\sqrt{N}}$ 

% \end{frame}
% \begin{frame}
%   \frametitle{Conclusion}
%   How many operations of G do we need though? \pause 
%   Clearly we require [$\frac{\pi}{4\theta}$] applications of G. As $\theta \ge sin(\theta) = \frac{1}{\sqrt{N}}$,
%   $$[\frac{\pi}{4\theta}] \le \frac{\pi \sqrt{N}}{4}$$ \pause 
%   Thus we need only $O(\sqrt{N})$ operations to achieve the task with probability $\ge \frac{\sqrt{N-1}}{\sqrt{N}} $

% \end{frame}
%---------------------------------------------------------


\end{document}s